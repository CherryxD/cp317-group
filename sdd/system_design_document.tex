\documentclass{article}

% Basic formatting
\usepackage[margin=0.7in]{geometry}
\usepackage[utf8]{inputenc}
\usepackage[yyyymmdd]{datetime}

\title{\textbf{Software Engineering Assignment SDD}}
\author{Trenton MacNinch, Michael Lin, and Ying Gao}

\begin{document}
\maketitle
\newpage

\tableofcontents
\newpage

\section{Introduction}
\subsection{Date of Issue}
\today
\subsection{Authorship}
Trenton MacNinch, Michael Lin, and Ying Gao
\subsection{Summary}
The program is used with the following syntax:

\texttt{calcgrades [name file] [course file] [output file]}
\newline \newline
The application reads two input text files and outputs one file with specific formatting.
The input files are given by the user, formatted as the following:
\texttt{ \newline
  [name file] \newline
  Student ID, Student Name \newline
  100000001, John Hay \newline
  100000002. Mary Smith \newline
  \ldots
}

\texttt{ \newline
  [course file] \newline
  Student ID, Course Code, Test 1, Test 2, Test 3, Final exam \newline
  100000001, CP317, 75.3, 80.4, 60.3, 70.5 \newline
  100000002, CP414, 80.2, 90.5, 50.4, 75.6 \newline
  100000001, CP414, 60.5, 70.6, 80.6, 75.6 \newline
  \ldots
}
\newline \newline
The output file is formatted as the following, with special notice given to student IDs being in ascending order:
\texttt{ \newline
  [output file] \newline
  Student ID, Student Name, Course Code, Final grade \newline
  1000000001, John Hay, CP317, 66.7 \newline
  1000000001, John Hay, CP414, 72.6 \newline
  1000000002, Mary Smith, CP414, 74.8 \newline
  \ldots
}

\section{Software Architecture}
\subsection{Overview}
Given the requirements for our application (Object oriented features, compiled language, file handling, etc.), we have chosen to create a C++ program running on a Linux system.
The overall system architecture then includes the following; 
\begin{itemize}
\item 2 Input Files
\item 1 Executable
\item 1 Output File
\end{itemize}

\section{Detailed Design}
\subsection{Overview}
The application loads all student information into a map, then loads all course information into a vector of custom course objects, and then sorts the vector of course objects. The program will then proceed to iterate through all stored course information and attach student names based on IDs while placing it in the designated output file. If the program encounters an error, it will print the encountered error.

\subsection{Student Information}
The program opens the name file as a normal C++ file handle. The program then calls a function that generates a map. This function reads the info from a line in the name file, validates the given information, and adds the student info to the map if the information is acceptable. 

\subsection{Course Information}
The program opens the course file as a file handle, and uses a function to generate a vector of course objects. The function reads the information from a line in the course file, validates the given course information, and rejects any invalid courses. The program then sorts the vector of course information by student ID to enable a more efficient outputting phase.

\subsection{Creating Output}
The program takes the information stored in the student id map and the course object vector, and combines it according to the output format while pushing it to the file handle for the output file. The program then closes.

\end{document}
